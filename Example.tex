% Example of monster5e class.
\documentclass{monster5e}

% While you can in principle write the stat blocks directly into your main .tex file,
% I find it better to write each monster card in a separate .tex file then \input it.
\newcommand{\ddir}{DnD_Monsters/}
% I define a custom Input command which does two things:
% 1) It automatically prepends the directory where my files live.
% 2) It removes the space which \input normally inserts after its inputs.
\newcommand{\Input}[1]{\input{\ddir #1}\unskip}

% This line points to where the symbol for the card background is.
\setBackground{MonsterBackground}

\begin{document}

% In creating a page of cards, there are a few things to do:
% Put the cards with no spaces between them.
% (You can have a space where there would be a line-break.)
\Input{Orc.tex}
\Input{Acolyte.tex}
\Input{Gray Ooze.tex}
\Input{Noble.tex}
\Input{Bandit.tex}
\Input{Adult Red Dragon.tex}

% If you want to print with backgrounds,
% every six cards you should enter the command \sixMonsterBackgrounds.

\sixMonsterBackgrounds
\end{document}